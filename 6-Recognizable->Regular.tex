\section{Recognizable implies regular}\label{sec:rec->reg}


\begin{theorem}\label{thm:Rec->Reg}
If a language of $\TWT$-graphs is recognizable, then it is regular.
\end{theorem}

We proceed gradually, by showing that this result holds for three sub-classes of $\TWT$ graphs. First for \emph{alternating-free domain-free graphs}, for \emph{domain-free} graphs (which we define below), then for domain graphs. We finally lift these results to the whole class of $\TWT$ graphs.  

\begin{definition}[Domain-free, alternation-free graphs]
A graph is \emph{domain-free} if all its domain modules are atomic. A graph is \emph{alternating} if it has a parallel module $P$ and a non-atomic series module $S$ such that either $P$ is a module of $S$ or $S$ is a module of $P$. It is \emph{alternation-free}  otherwise.  
\end{definition}

 In all these steps, we use the following lemma:
\begin{lemma}\label{lem:imposing-type-is-recognizable}
Let $L$ be a language of $\TWT$-graphs, $x$ a letter and $\tau$ a type. If $L$ is recognizable then the restriction of $L$ to the graphs of type $\tau$ (resp.  the graphs where $x$ is $\tau$-guarded, word graphs, multiset graphs, domain-free graphs, alternation-free graphs) is also recognizable.
\end{lemma}
%\begin{proof}
%
%\end{proof}

\subsection{Alternation-free domain-free graphs}

\begin{lemma}\label{lem:alt-free-dom-free-description}
If $G$ is an alternation-free domain-free $\TWT$ graph, then $G=H[\vec{T}/\vec{x}]$ where 
\begin{itemize}
\item $H$ is either a word or a multiset graph,
\item $\vec{T}$ are multiset graphs of type test not containing the letters $\vec{x}$ and 
\item  the  letters of $\vec{x}$ are $\mathsf{t}$-guarded in $G$.
\end{itemize}  
\end{lemma}
%\begin{proof}
%\todo{complete}
%\end{proof}

\begin{proposition}\label{prop:rec->reg-alt-free-dom-free}
If a language of alternation-free domain-free $\TWT$ graphs is recognizable, then it is regular.
\end{proposition}
\begin{proof}
Let $L$ be a language of non-alternating domain-free graphs, $\A$ an algebra of domain $D$, $h:\mathbb{G}_\TWT(\Sigma) \to \A$ a homomorphism and $F\subseteq D$ such that $h^{-1}(F)=L$. We denote by $L_v$ the set of graphs whose image by $h$ is $v$. Note that we have the following equality:
 $$L\ =\ \underset{f\in F}{\cup} L_f.$$ 
We show in the following that $L_v$ is regular for every $v\in D$, and this is enough to conclude. 
\medskip

For every $v\in D$, we associate a new letter $x_v$, and denote this new set of letters by $\Gamma$.  We extend the homomorphism $h$ to $\TWT$-graphs over the alphabet $\Sigma \cup \Gamma$ by letting $h(x_v)=v$ for every $x_v\in\Gamma$. 

For every $v\in D$, let $T_{v}$ be the set of multiset graphs over $\Sigma$ of type test  whose image by $h$ is $v$, and let $M_{v}$ be the set of word or multiset graphs over  $\Sigma\cup \Gamma$, where the letters of $\Gamma$ are $\mathsf{t}$-guarded.  Using Lem.~\ref{lem:alt-free-dom-free-description}, we have:
$$ L_v=M_v[T_w/x_w,\  w\in D]$$ 
By Lem.~\ref{lem:imposing-type-is-recognizable} and using the fact that recognizability implies regularity for word and multiset graphs, we conclude that $L_v$ is regular for every $v\in D$.
\end{proof}

\subsection{Domain-free graphs}
 
\begin{definition}[Guarded pure modules]
Let $G$ be a domain-free graph and $M$ a module of $G$. We say that $M$ is guarded if either $M$ is a maximal parallel module, or $M$ is a non-atomic series module, such that there is a module of $G$ which is parallel to $M$.
\end{definition}

\begin{lemma}\label{lem:no-guarded-modules-imply-alt-free}
If a domain-free graph has no guarded modules, then it is alternation-free.
\end{lemma}

\begin{proposition}\label{prop:rec->reg-dom-free}
If a language of domain-free graphs is recognizable, then it is regular.
\end{proposition}
\begin{proof}
Let $L$ be a language of domain-free graphs, $\A$ an algebra of domain $D$, $h:\mathbb{G}_\TWT(\Sigma) \to \A$ a homomorphism and $F\subseteq D$ such that $h^{-1}(F)=L$. Let us show that $L_v$, the set of graphs over $\Sigma$ whose image (by $h$) is $v$,  is regular for every $v\in D$. 
\medskip

We associate every $v\in D$ with two new letters $x_v$ and $y_v$, and let  $\Gamma:=\set{x_v\ | \ v\in D}$ and $\Delta:=\set{y_v\ | \ v\in D}$. If $Q\subseteq D$, we denote by $X_D$ and $Y_D$ the subsets of $\Gamma$ and $\Delta$ corresponding to these elements.   We extend the homomorphism $h$ to $\TWT$ graphs over the alphabet $\Sigma \cup \Gamma \cup \Delta$ by letting $h(x_v)=h(y_v)=v$ for every $x_v\in\Gamma$ and $y_v\in\Delta$.
\medskip

Let $v\in D$, $Q, R\subseteq D$, $X\subseteq \Gamma$, and $Y\subseteq \Delta$.  We define the set of graphs $L^{Q,R,X,Y}_{v}$ as follows. A graph  $G$ is in this set if and only if:

\begin{itemize}
\item $G$ is a domain-free graph over the alphabet $\Sigma\cup X\cup Y$, 
\item the image of $G$ is $v$,
\item  the image of the strict guarded series modules of $G$ belong to $Q$, 
\item   the image of the strict guarded parallel modules of $G$ belong to $R$, 
\item the letters of $X$ are $\mathsf{s}$-guarded  in $G$ ,
\item the letters of $Y$ are $\mathsf{p}$-guarded in $G$.  
\end{itemize}
\smallskip

 Let $S^{Q,R,X,Y}_{v}$ and $P^{Q,R,X,Y}_{v}$ be the restriction of $L^{Q,R,X,Y}_{v}$ to series and parallel graphs respectively. Let us show that these two languages are regular  when $X\cap X_Q=\emptyset$ and $Y\cap Y_Q=\emptyset$. We proceed by induction on the size of $Q\cup R$. When $Q=R=\emptyset$, the graphs of these sets  alternation-free graphs thanks to  Lemma~\ref{lem:no-guarded-modules-imply-alt-free}. Using Lemma~\ref{lem:imposing-type-is-recognizable} and Prop~\ref{prop:rec->reg-alt-free-dom-free}, we conclude the base case. Let us show the inductive case. We have the following equality:
\begin{align*}
S^{Q\cup\set{w},R,X,Y}_v=&S^{Q,R,X\cup\set{x_w},Y}_v[\mu x_w. S^{Q,R,X\cup\set{x_w},Y}_w/x_w][S^{Q,R,X,Y}_w/x_w]\\
S^{Q,R\cup\set{w},X,Y}_v=&S^{Q,R,X,Y\cup\set{y_w}}_v[\mu y_w. P^{Q,R,X,Y\cup\set{y_w}}_w/y_w][P^{Q,R,X,Y}_w/y_w]
\end{align*}
We use similar equations to handle the case of parallel graphs.
\medskip

\noindent Finally, notice that for every $v\in D$, we have:
$$ L_v=L^{\emptyset,\emptyset,\Gamma,\Delta}_v[S^{D,D,\emptyset,\emptyset}_w/x_w, w\in D] [S^{D,D,\emptyset,\emptyset}_w/y_w, w\in D]$$
The graphs of the set $L^{\emptyset,\emptyset,\Gamma,\Delta}_v$ are alternation-free and domain-free, its regularity follows from Lemma~\ref{lem:imposing-type-is-recognizable} and Prop.~\ref{prop:rec->reg-alt-free-dom-free}.
\end{proof}

\subsection{Domain graphs}

\begin{lemma}
Let $G$ be a domain graph whose domain modules, distinct from $G$ itself, are all atomic. There is a domain-free graph $H$ such that $G=\dom(H)$.   
\end{lemma}

\begin{proposition}
If a language of domain graphs is recognizable, then it is regular.
\end{proposition}
\begin{proof}
Let $L$ be a language of domain graphs, $\A$ an algebra whose domain is $D$, $h:\mathbb{G}_\TWT(\Sigma) \to \A$ a homomorphism and $F\subseteq D$ such that $h^{-1}(F)=L$. Let us show that $L_v$, the set of graphs over $\Sigma$ whose image  is $v$,  is regular for every $v\in D$. 
\medskip

We associate every $v\in D$ with a new letter $x_v$ and let  $\Gamma:=\set{x_v\ | \ v\in D}$. If $Q\subseteq D$, we denote by $X_D$ the letters of $\Gamma$ corresponding to these elements.   We extend the homomorphism $h$ to $\TWT$-graphs over the alphabet $\Sigma \cup \Gamma$ by letting $h(x_v)=v$ for every $x_v\in\Gamma$. 
\medskip

Let $v\in D$, $Q\subseteq D$ and $X\subseteq \Gamma$.  We define the set of graphs $L^{Q,X}_{v}$ as follows. We let $G$ be in this set  if and only if:
\begin{itemize}
\item $G$ is a domain graph over the alphabet $\Sigma\cup X$, 
\item the image of $G$ is $v$,
\item  the image of the strict domain modules of $G$ belong to $Q$.
\end{itemize}
\medskip

\noindent Let us show that $L^{Q,X}_v$ is regular. We proceed by induction on the size of $Q$. Suppose that $Q=\emptyset$. For every $w\in D$, let $M^X_w$ be the set of domain-free graphs over the alphabet $\Sigma\cup X$ whose image is $w$.  
 We have the following equation:
$$ L^{\emptyset, X}_v= \bigcup_{\substack{w\in D\\ \dom(w)=v}}\dom(M_w)$$
 which concludes the base case. To handle the inductive case, we notice the following equality:
 $$L^{Q\cup{w}, X}_v= L^{Q, X\cup\set{x_w}}_v[\mu x_w L^{Q, X\cup\set{x_w}}_w/x_w][L^{Q, X}_w/x_w]$$
\end{proof}

\subsection{$\mathsf{TW}_2$ graphs}

\begin{lemma}\label{structure-of-twt-graphs}
If $G$ is a $\TWT$-graph, then $G=H[\vec{D}/\vec{x}]$ where $H$ is domain-free and $\vec{D}$ are domain graphs. 
\end{lemma}

Now we are ready to prove Thm.~\ref{thm:Rec->Reg}.
\begin{proof}[Proof of Thm.~\ref{thm:Rec->Reg}]
Let $L$ be a language of domain graphs, $\A$ an algebra whose domain is $D$, $h:\mathbb{G}_\TWT(\Sigma) \to \A$ a homomorphism and $F\subseteq D$ such that $h^{-1}(F)=L$. Let us show that $L_v$, the set of graphs over $\Sigma$ whose image  is $v$,  is regular for every $v\in D$. 
\medskip

We associate every $v\in D$ with a new letter $x_v$ and let  $\Gamma:=\set{x_v\ | \ v\in D}$.   We extend  $h$ to $\TWT$-graphs over the alphabet $\Sigma \cup \Gamma$ by letting $h(x_v)=v$ for every $x_v\in\Gamma$. 
\medskip

For every $v\in D$, we let $M_v$ be the set of domain-free graphs over the alphabet $\Sigma\cup \Gamma$ whose image is $v$, and $N_v$ be the set of domain graphs over the alphabet $\Gamma$ whose image is $v$. We conclude by noticing the following equation, consequence of Lem.~\ref{structure-of-twt-graphs}:
$$ L_v=M_v[N_w/x_w, w\in D]$$
\end{proof}
\begin{remark}
By analyzing this proof, we see that we never use iteration over test graphs. 
\end{remark}